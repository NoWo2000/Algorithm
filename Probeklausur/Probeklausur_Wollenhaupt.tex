\documentclass[12px,a4paper]{article}
\usepackage[utf8]{inputenc}
\usepackage{ngerman}
\usepackage[document]{ragged2e}
\usepackage{enumerate}
\usepackage{amssymb}


\begin{document}
\begin{titlepage}
	\centering
	{\scshape\LARGE DHBW Manhheim \par}
	\vspace{1cm}
	{\scshape\Large 2. Semester Cyber Security\par}
	\vspace{1.5cm}
	{\huge\bfseries Algorithmen und Komplexität\par}
	\vspace{2cm}
	{\Large\itshape N.W. \& J.T\par}
	\vfill

% Bottom of the page
	{\large \today\par} 
\end{titlepage}
\newpage
\justify

\section*{Aufgabe Nr. 1}
\subsection*{Eigenschaften der Groß-O-Notation}
\begin{enumerate}
	\item Geben Sie die Definition der $\mathcal{O}$-Notation an.
	\item Es sei $f \in \mathcal{O}(h_1)$ und $g \in \mathcal{O}(h_2)$. Zeigen Sie, dass $f \cdot g \in \mathcal{O}(h_1 \cdot h_2)$ gilt.
	\item Beweisen Sie, dass für alle $f: \mathbb{N} \rightarrow \mathbb{R}_+$ gilt, dass $f \in \mathcal{O}(f)$.
	\item Beweisen Sie, dass für $f, \: g: \mathbb{N} \rightarrow \mathbb{R}_+$ und $d \in \mathbb{R}_+$ gilt, dass \\ \noindent\hspace*{5mm} $g \in \mathcal{O}(f) \rightarrow d \cdot g \in \mathcal{O}(f)$.
	\item Beweisen Sie, dass für $f, \: g, \: h: \mathbb{N} \rightarrow \mathbb{R}_+$ gilt, dass \\ \noindent\hspace*{5mm} $f \in \mathcal{O}(h) \ \land g \in \mathcal{O}(h) \rightarrow f+g \in \mathcal{O}(h)$.
	\item Beweisen Sie, dass für $f, \: g, \: h: \mathbb{N} \rightarrow \mathbb{R}_+$ gilt, dass \\ \noindent\hspace*{5mm} $f \in \mathcal{O}(g) \ \land g \in \mathcal{O}(h) \rightarrow f \in \mathcal{O}(h)$.
	\item Angenommen $f, \: g, \: h: \mathbb{N} \rightarrow \mathbb{R}_+$. Außerdem wird angenommen, dass der Grenzwert von  \noindent\hspace*{5mm}  $$\lim\limits_{n \rightarrow \infty} \frac{f(n)}{g(n)}$$ existiert. Beweisen sie, dass dann auch $f \in \mathcal{O}(g)$ gilt.
	\item Es seien $f, \: g \in \mathbb{R}_+$. Geben Sie die Definition $f \sim g$ an.
\end{enumerate}

\newpage
\section*{Aufgabe Nr. 2}
\subsection*{Groß-O-Notation}
\begin{enumerate}
	\item Zeigen Sie, dass $n^2 \in \mathcal{O}(2^n)$ ist.
	\item Zeigen Sie auch, dass $n^3 \mathcal{O}(2^n)$ gilt.
	\item Zeigen Sie: $ \log_2(n) \in \mathcal{O}(\ln(n+1)) $
	
	\item Zeigen Sie, dass $\ln^2 (n) \in \mathcal{O}(\sqrt{n})$ gilt.
	\item Versuchen Sie zu zeigen, dass $n^{\alpha} \in \mathcal{O}(2^n)$, wenn angenommen werden kann, dass $\alpha \in \mathbb{N}$ vorausgesetzt ist.
\end{enumerate}
\newpage
\section*{Aufgabe Nr. 3}
\subsection*{Rekurrenzgleichungen}
Lösen Sie folgende Rekurrenzgleichungen: \\
\begin{enumerate}
	\item $x_{n+2}=x_{n+1} + x_n$ für welche gilt:	$x_0 = 0$ und $x_1 =1$
	\item $x_{n+2}=4 \cdot x_{n+1} - 4 \cdot x_n +1$ für welche gilt:	$x_0=1$ und $x_1=3$
	\item $a_{n+2} = \frac{1}{6} \cdot a_{n+1} + \frac{1}{6} \cdot a_n$ für welche gilt:	$a_0 = 0$ und $a_1 = \frac{5}{6}$
	\item $a_{n+2} = -\frac{1}{2} \cdot a_{n+1} + \frac{1}{2} \cdot a_n$ für welche gilt: $a_0 = 2$ und $a_1 = 1$
	\item $a_{n+2} = a_{n+1} + 2 \cdot a_n +1$ für welche gilt:	$a_0 = 0$ und $a_1 = - \frac{1}{2}$
	\item $a_{n+2} = a_n +2$ für welche gilt:	$a_0 = 2$ und $a_1 = 1$
	\item $a_{n+2} = 2 \cdot a_n - a_{n+1}$ für welche gilt:	$a_0 = 0$ und $a_0 = 3$
	\item $a_{n+2} = 7 \cdot a_{n+1} - 10 \cdot a_n$ für welche gilt:	$a_0 = 0$ und $a_0 = 3$
	\item $a_{n+1} = 2^n \cdot a_n$ für welche gilt: $a_1 = 1$
	\item Stellen Sie mit dem Ansatz $a_k := f(2^k)$ eine Rekurrenzgleichung auf und lösen Sie diese.\\ 
			\noindent\hspace*{5mm} $f(n) = 2 \cdot f(n\backslash 2) + \log_2(n)$ \\
			Es gelten folgende Anfangsbedingungen:	$x_0 = 0$ und $x_1 = 1$
\end{enumerate}
\newpage
\section*{Aufgabe Nr. 4}
\subsection*{Master Theorem}
\begin{enumerate}
	\item Geben Sie die Definition des Master-Theorems an.
	\item Schätzen Sie mit Hilfe des Master-Theorems die Komplexität von f ab. \\
	\noindent\hspace*{0.5mm} $f(n) = 2 \cdot f(n\backslash 2) + n$
	\item Schätzen Sie $g(n) = 4 \cdot g(n\backslash 3) + ( \frac{2}{3} ) ^2 \cdot n$ mit Hilfe des Master-Theorems ab.
	\item Schätzen Sie $g(n) = 4 \cdot g(n\backslash 5) + ( \frac{3}{2} ) ^3 \cdot n^2$ mit Hilfe des Master-Theorems ab.
	\item Schätzen Sie $g(n) = 4 \cdot g(n\backslash 3) + 2 \cdot n^{\log_3(4)} + n$ mit Hilfe des Master-Theorems ab.
\end{enumerate}
\newpage
\section*{Aufgabe Nr. 5}
\begin{enumerate}
	\item Geben Sie die Gleichung an, mit der wir Merge Sort definiert haben. Alternativ ist auch der Pythoncode von Merge Sort akzeptabel.
	\item Wir haben die Funktion, die für zwei sortierte Listen $L_1, L_2$ berechnet, wie viele Vergleichsoperationen beim Aufruf von $merge(L_1, L2)$ geschehen, mit $cmpCount$ bezeichnet. geben Sie die Ungleichung an, die wir für das Verhältnis zwischen $\#L_1, \#L_2$ und $cmpCount(L_1, L_2)$ aufgestellt haben. 
	\item Schätzen Sie mit Hilfe des Master-Theorems die Komplexität von $Merge Sort$ ab.
\end{enumerate}
\newpage
\section*{Aufgabe Nr. 6}
\subsection*{Sortierproblem}
\begin{enumerate}
	\item Definieren Sie:	
	\begin{enumerate} 
	\item Partielle Ordnung
	\item Lineare Ordnung
	\item Quasiordnung
	\item Totale-Quasiordnung
	\end{enumerate}
	
	\item Geben Sie die Definition des Sortierproblems an.
\end{enumerate}
\newpage
\section*{Aufgabe Nr. 7}
\subsection*{Sortieralgorithmen}
\begin{enumerate}
	\item Insertion Sort
	\begin{enumerate}
	\item Formale Defintion
	\item Implementierung 
	\item Komplexität
	\end{enumerate}
	\item Selection Sort
	\begin{enumerate}
	\item Formale Defintion
	\item Implementierung 
	\item Komplexität
	\end{enumerate}
	\item Merge Sort
	\begin{enumerate}
	\item Formale Defintion
	\item Implementierung 
	\item Komplexität
	\item Komplexität im Master-Theorem
	\end{enumerate}
	\item Quicksort
	\begin{enumerate}
	\item Formale Defintion
	\item Implementierung 
	\item Komplexität
	\end{enumerate}
	\item Counting Sort
	\begin{enumerate}
	\item Formale Defintion mit allen Phasen
	\item Implementierung 
	\item Komplexität
	\end{enumerate}
	\item Radix Sort
	\begin{enumerate}
	\item Vorgehensweise
	\item Implementierung 
	\end{enumerate}
	\item Heapsort
	\begin{enumerate}
	\item Formale Defintion
	\item Implementierung 
	\item Komplexität
	\end{enumerate}
\end{enumerate}
\newpage

\section*{Aufgabe Nr. 8}
\subsection*{ADTs + Set und Maps}
\begin{enumerate}
	\item Geben Sie die formale Definition von ADTs an.
	\item Geben Sie die formale Definition des ADT Stack an.
	\item Beschreiben Sie was ein Generator in einem 
	\item Beschreiben Sie den Shunting-Yard-Algorithmus.
	\item Geben Sie die formale Definition des ADT Map an.
	\item Was haben geordnete Binärbäume, AVL Bäume, Tries und co. miteinander zu tun, was verbindet sie?
	\item Geben Sie die formale Definition und die Komplexität von geordneten Binärbäumen an.
	\item Geben Sie die formale Definition und die Komplexität von AVL Bäumen an.
	\item Geben Sie die formale Definition und die Komplexität von Tries an.
	\item Beschreiben Sie, was eine Prioritätswarteschlange ist. 
	\item Geben Sie die formale Definition und die Komplexität von Prioritätswarteschlangen an.
	\item Geben Sie die formale Definition und die Komplexität des Heaps an.
\end{enumerate}


\end{document}































